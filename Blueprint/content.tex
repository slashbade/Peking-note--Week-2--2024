\section{Group actions}

\begin{defi}
    A (left) group action of a group $G$ on a set $A$ is a map from $G \times A$ to $A$ such that $(g,a) \mapsto g\cdot a = ga$ satisfying the following properties:
    \begin{enumerate}
        \item $g_1\cdot (g_2 \cdot a) = (g_1g_2)\cdot a$ for any $g_1, g_2 \in G$ and $a \in A$
        \item $1 \cdot a = a$ for any $a \in A$.
    \end{enumerate}
\end{defi}

\begin{example}
    \begin{enumerate}
        \item We have the natural action of $S_n$ on $\{1, \dots, n\}$.
        \item The multiplication map $G \times G \rightarrow G$ defines an action of $G$ on itself.
    \end{enumerate}
\end{example}


\begin{prop}
    Let a group $G$ act on a set $A$. Then we have a group homomorphism
    \[
        G \rightarrow Perm(A), \quad g \mapsto (a \mapsto g\cdots a).
    \]
\end{prop}
\begin{proof}
    Let us write $\sigma_g$ for the map $\sigma_g: A \rightarrow A$, $\sigma_g(a) = g\cdot a$.

    We first check this is a bijection, hence a well-defined map from $G$ to $Perm(A)$. We then check this is a group homomorphism.
\end{proof}
\begin{remark}
    It follows that a group action on a set is equivalent to a group homomorphism to the permutation group.
\end{remark}

\begin{example}
    Let $G$ be a group and $A$ be a set. We always have the trivial action of $G$ on $A$, that is, $g\cdot a =a$ for any $g\in G$ and $a\in A$.
\end{example}

\begin{example}
    Let $G$ be a group. We define the conjugation action of $G$ on its own by $g\cdot h = ghg^{-1}$.
    That is, for each $g \in G$, define $c_g \colon G \to G$
    to be conjugation
    $$
        c_g(x)=gxg^{-1}.
    $$

    We show that it is an action.
    To verify axiom (1), note that for each $x \in G$,
    \begin{align*}
        (c_g \circ c_h)(x) = & c_g(c_h(x))       \\
        =                    & c_g(hxh^{-1})     \\
        =                    & g(hxh^{-1})g^{-1} \\
        =                    & (gh)x(gh)^{-1}    \\
        =                    & c_gh(x).
    \end{align*}
    Therefore, $c_g \circ c_h = c_{gh}$.
    To prove axiom (2), note that for each $x \in G$,
    $c_1(x) = 1x1^{-1} = x$.
\end{example}


\begin{example}
    Let $H$ be any subgroup of $G$.
    Define an action of $G$ on $G/H$ by the left translation
    $$
        \tau_g\colon  aH \mapsto gaH \text{ for all } g\in G,~ aH\in G/H.
    $$
    This satisfies the two axioms for a group action.
    Also, $\tau_g$ is a permutation in $S_{G/H}$ and the map $g\mapsto \tau_g$ is a homomorphism from $G$ to $S_{G/H}$.
\end{example}


\begin{defi}[Stabilizers]
    Let a group $G$ act on a set $A$.
    \begin{enumerate}
        \item For any $a \in A$, we define the stabilizer subgroup of $G$ by
              \[
                  G_a = {\rm Stab}_G(a) = \{g \in G \vert g \cdot a = a\}.
              \]

        \item For any subset $B \subset A$, we define
              \[
                  {\rm Stab}_G(B) = \cap_{a\in B} {\rm Stab}_G(a) = \{g \in G \vert g\cdot a = a \forall a \in B\}.\]

        \item We define the kernel of the action by ${\rm Stab}_G(A)$.
    \end{enumerate}
\end{defi}

\begin{lem}Both $G_a$ and $ {\rm Stab}_G(B)$ are subgroups of $G$. The subgroup ${\rm Stab}_G(A)$ is the kernel of the corresponding group homomorphism $G \rightarrow Perm(A)$ of the group action.
\end{lem}

\begin{proof}

\end{proof}

\begin{defi}
    \begin{enumerate}
        \item Let $A \subset G$ be a non-empty subset of $G$. Define  $C_G(A) = \{g \in G \vert g a g^{-1} = a \text{ for all } a \in A\}$. This subgroup is called the centralizer of $A$ in $G$. (Check this is indeed a subgroup.)
        \item The center of $G$ is defined to be the subgroup $Z(G) = \{g\in G \vert g a g^{-1} =a \text{ for all } a \in G\}$. (Check this is indeed a subgroup.)
        \item Let $A \subset G$ be a non-empty subset of $G$. The normalization of $A$ is defined to be $N_G(A) = \{g \in G \vert gAg^{-1} = A\}$. (Check this is indeed a subgroup.)
    \end{enumerate}
\end{defi}

\begin{lem}
    \begin{enumerate}
        \item We consider the action of $G$ on itself by conjugation, that is, $g \cdot a = gag^{-1}$. The $Stab_G(A) = C_G(A)$ for any $A \subset G$.
        \item Let $\mathcal{P}(G)$ be the power set of $G$. We consider the action of $G$ on $\mathcal{P}(G)$ by conjugation. Then $N_G(A) = Stab_G(A)$ for any $A \subset G$.
    \end{enumerate}
\end{lem}

\begin{example}
    \begin{enumerate}
        \item Let $G$ be an abelian group. Then $C_G(A) = N_G(A) = Z(G) = G$ for any subset $A \subset G$.
        \item Let $S_4$ acts on $\{1,2,3,4\}$ in the natural way. Then we have $Stab_G(4) = S_3$ and the kernel of this action is $\{e\}$.
        \item Let $(12) \in S_4$. We compute $C_{S_4}((12)) = \{e, (12), (34), (12)(34)\}$.

              We have $N_{S_4}((12)) = C_{S_4}((12))$.


    \end{enumerate}
\end{example}

\subsection{Orbits}
\begin{defi}
    Let $G$ act on a set $A$. Let $a \in A$. The orbit of $a$ is defined as $\mathcal{O}(a) = G \cdot a = \{g a \in A \vert g \in G\}$.

    We say the action of $G$ on $A$ is transitive if $A = G \cdot a$ for some $a \in A$.
\end{defi}

\begin{example}
    \begin{enumerate}
        \item The left multiplication of $G$ on itself is transitive.
        \item The natural action of $S_n$ on $\{1,2, \dots, n\}$ is transitive.
        \item The conjugation action of $S_3$ on itself has $3$ orbits.
    \end{enumerate}
\end{example}

\begin{lem}Let $G$ act on a set $A$.
    \begin{enumerate}
        \item For any two orbits $\CO(a)$ and $\CO(b)$, we have either  $\CO(a) = \CO(b)$ or $\CO(a) \cap \CO(b) = \emptyset$. Therefore we have a parition of $A$ by orbits.
        \item For any $a \in A$, we have bijection between the set of cosets $G/ Stab_{G}(a)$ and the $\CO(a)$ orbit of $a$.
        \item Assume $G$ is a finite group. Then the cardinality of any orbit must divide $|G|$.
    \end{enumerate}
\end{lem}

\begin{proof}
\end{proof}

\begin{cor}
    Let $G$ act on a finite set $A$. Let $I \subset A$ be a set of representative of $G$-orbits. Then we have
    \[
        | A | = \sum_{a \in I} | \mathcal{O}(a)|.
    \]
\end{cor}


\subsection{More on group actions}



\begin{thm}Let $G$ be a finite group and $H \le G$ be a subgroup of $G$. Then the order of $H$ divides the order of $G$ and the number of left cosets of $H$ in $G$ equals $|G| / |H|$.

    In particular, we have $|H| \mid |G|$ if $|G|$ is finite.
\end{thm}

\begin{proof}

\end{proof}

\begin{defi}
    Let $G$ be a (potentially infinite) group with a subgroup $H$. The number of left cosets of $H$ in $G$ is called the index of $H$ in $G$ and is denoted by $|G : H|$.
\end{defi}

\begin{example}
    We have $|\BZ : 2\BZ| =2$. Note that both $\BZ$ and $2\BZ$ are infinite.
\end{example}

\begin{example}
    \begin{enumerate}
        \item We consider the conjugation action of $G$ on $G$. Then we have $Stab_G(G) = G$ and  $\cap_{g \in G} Stab_G(a) = Z(G)$.
        \item We consider the action $G$ on $G/H$ via left multiplication. This action is transitive. We have $Stab_G(H) = H$. However, the kernel of this action is $\cap_{g \in G} g H g^{-1}$.
    \end{enumerate}
\end{example}


\begin{thm}[Cayley's theorem]
    Any group is isomorphic to a subgroup of some permutation group. If $G$ is finite of order $n$, then $G$ is isomorphic to a subgroup of $S_n$.
\end{thm}
\begin{proof}
\end{proof}

\begin{prop}
    Let $G$ be a finite group of order $n$. Let $p$ be the smallest prime factor of $n$. Then any subgroup of index $p$ is normal (provided such a subgroup exists).
\end{prop}

\begin{proof}
    Let $H$ be a subgroup of $G$ with index $p$. We consider the action of $G$ on $G/H$. Let $K = \cap_{g \in G} g H g^{-1} \subset H$ be the kernel of this action. Then we have a group homomorphism $\phi: G \rightarrow S_p$ such that $G/K \cong \phi(G)$ by the first isomorphism theorem.

    We see that $|G/K| = | \phi(G)|$ must be a factor of $|S_p| = p!$. We have $n =|G| = |K| | \phi(G)|$. Since the smallest prime factor of $n$ is $p$. We can only have $|G/K| = p$, or $|K| = n/p = |H|$. We have $K= H$.
\end{proof}

\begin{cor}
    Let $G$ be a finite group. Then any subgroup of index $2$ must be normal.
\end{cor}


\subsection{Conjugacy classes}
\begin{defi}
    The orbits of $G$ acting on itself by conjugation is called conjugacy classes of $G$.
\end{defi}

\begin{example}
    \begin{enumerate}
        \item Let $G$ be abelian. Then each conjugacy class consists of a single element of $G$.
        \item The group $S_3$ has three conjugacy classes.
        \item Let $z \in Z(G)$. Then the conjugacy of $z$ is precisely $\{z\}$.
    \end{enumerate}
\end{example}


\begin{prop}
    Let $G$ be a finite group and let $g_1, \dots, g_n$ be representatives of conjugacy classes of $G$ not contained in the center. Then we have
    \[
        | G|  = |Z(G)| + \sum_{i=1}^n |G : C_{G}(g_i)|.
    \]
\end{prop}
\begin{proof}
\end{proof}
\begin{cor}
    Let $G$ be a group of order $p^n$ for some prime $p$. Then $Z(G)$ is non-trivial.
\end{cor}

\begin{proof}
    We know $|Z(G)| \ge 1$, since the identity element is in the center. Recall the class equation:
    \[
        | G| - \sum_{i=1}^n |G : C_{G}(g_i)| = |Z(G)|.
    \]

    Note that $|G : C_{G}(g_i)| > 1$, since $C_{G}(g_i) \neq G$ by definition. Therefore $p \mid |Z(G)|$. Since $|Z(G)| \neq 0$, we must have $|Z(G)| >1$. This finishes the proof.
\end{proof}

Let us next give an explicit description of conjugacy classes of the symmetric $S_n$.

\begin{defi}
    Let $n$ be positive integer. A partition of $n$, denoted by $\lambda  \vdash n$, is a nondecreasing sequence $\lambda = (\lambda_1, \dots, \lambda_k)$ of positive intergers such that $\sum \lambda_i = n$. We denote the set of partitions of $n$ by $\CP(n)$.
\end{defi}

\begin{thm}
    The set of conjugacy classes of $S_n$ is in natural bijection with $\CP(n)$.
\end{thm}

\begin{proof}

\end{proof}

\subsection{Subgroups of cyclic groups}
\begin{defi}
    A group $G$ is called cyclic if $G$ can be generated by a single element, i.e., $G = \langle x\rangle$ for some $x \in G$.
\end{defi}
Let $G$ be an arbitrary group and $x \in G$. Then  the  subgroup $\langle x\rangle$ generated by $x$ is a cyclic group. So we are studying the easieast subgroups of a group $G$.

Let $G = \langle x \rangle$ be a cyclic group throughout this section.
\begin{lem}
    Let $G = \langle x \rangle $. Then $|G| =ord(x)$.
\end{lem}
\begin{proof}

\end{proof}

\begin{cor}
    If $|G| = n$, then we have $ G \cong \BZ/n\BZ$. If $|G| = \infty$, then we have $ G \cong \BZ$.
\end{cor}
\begin{example}
    \begin{enumerate}
        \item For any $n \in \BZ$, the quotient group $\BZ/ n \BZ$ is cyclic. We can take $\overline{1}$ as the cyclic generator.
        \item The group $S_3$ is NOT cyclic.
    \end{enumerate}
\end{example}

\begin{lem}
    Let $p \in \BZ$ be a prime. If $G$ is a group of order $p$, then $G$ is isomorphic to the cyclic group $\BZ/p\BZ$.

    In other words, there is a unique group of order $p$ up to isomorphism.
\end{lem}

\begin{prop}
    Let $H \le G$ be a subgroup. Then $H = \langle x^a \rangle$ for some $a \in \BZ$ is also cyclic.
\end{prop}

\begin{proof}

\end{proof}

\begin{cor}
    Let $H= \langle x^a \rangle $ be a subgroup of $G$. Let $d \ge 0$ be the g.c.d. of $a$ and $n$. Then $H =  \langle x^d \rangle $.
\end{cor}
\begin{proof}

\end{proof}

\begin{cor}
    Let $H= \langle x^d \rangle $ be a subgroup of $G$ such that $d \ge 0$ and  $d \mid n$. Then $|H| =n/d$.
\end{cor}
\begin{proof}

\end{proof}


We summarize the discussion above as the following theorem.

\begin{thm}
    Let $G = \langle x \rangle$ be a cyclic group of order $n$.
    Then $\{\langle x^d \rangle \vert d \ge 0, d \mid n\}$ is the set of all non-identical subgroups of $G$.
\end{thm}

\begin{proof}

\end{proof}

\begin{prop}
    Let $H_1 = \langle x^{d_1} \rangle$ and $H_2 = \langle x^{d_2} \rangle$ be subgroups of $G$ with $d_i \ge 0$ and $d_i \mid n$. Then we have
    \[
        H_1 \cap H_2 = \langle x^{s} \rangle, \qquad \langle H_1 \cup H_2 \rangle = x^{t}.
    \]
    Here $t = gcd(d_1, d_2)$ and $t = lcm (d_1, d_2)$.
\end{prop}

\subsection{Automorphisms of cyclic groups}

Let $G = \langle x \rangle$ be a cyclic group of order $n$. Recall the ring $\BZ/n\BZ$.
\begin{lem}
    Let $End(G)$ be the set endomorphisms of $G$, i.e., group homomorphisms from $G$ to $G$. We have a bijection
    \[
        End(G) \cong \BZ/n\BZ, \sigma \mapsto a(\sigma) \quad \text{ such that } \sigma \circ \sigma' \mapsto  a(\sigma) a(\sigma').
    \]
\end{lem}
\begin{proof}
\end{proof}

Let $Aut(G)$ be the automorphism group of $G$.
\begin{thm}
    We have a group isomorphism
    \[
        Aut(G) \cong (\BZ/n\BZ)^* = \{\overline{a} \in \BZ/n\BZ\vert gcd(a,n) =1\}
    \]
\end{thm}
\begin{proof}

\end{proof}

As we have seen in the proof, to understand the precise structure of the group $Aut(G)$ we need to understand the ring $\BZ/n\BZ$. This will be the topic for the future semester.





\section{Sylow theorems I}
\begin{defi}
    Let $G$ be a finite group and let $p$ be a prime.

    \begin{enumerate}
        \item A group of order $p^n (n > 0)$ is called a $p$-group. Subgroups of $G$ of order $p^n$ is called $p$-subgroups.
        \item Assume $|G| = p^n m$ with $p \nmid m$. Then a subgroup of $G$ of order $p^n$ is called a Sylow $p$-subgroup of $G$.
        \item The set of all Sylow $p$-subgroups of $G$ is denoted by $Syl_p(G)$. We denote the cardinality of $Syl_p(G)$ by $n_p=n_p(G)$.
    \end{enumerate}
\end{defi}

\begin{lem}
    Let $G$ be a finite abelian group and let $p$ be a prime that divides the order of $G$. Then $G$ contains an element of order $p$.
\end{lem}

\begin{proof}
    We proceed by induction on $|G|= p^{n}m$. Let $x \in G$ be a non-trivial element, and write $\langle x \rangle = H$. Then $H$ is not trivial by assumption. If $H= G$, then we can take $P= \langle x^{p^{n-1}m} \rangle $. If $H \neq G$ with $p \vert |H|$, we can proceed with induction hypothesis. In any case, if $p \vert |H|$, we are done.

    So we can assume $p \nmid |H|$. Note that since $G$ is abelian, $H$ is normal. We have $p \mid |G/H|$ and $1 < |G/H| < |G|$. By induction hypothesis, we can find $yH \in G/H$ of order $p$. This means
    \[
        y^p = h \in H, \text{with } y \neq e.
    \]
    In particular, we have $y^a \notin H$ for any $a$ coprime to $p$. Let $ord(h) = a$. Since $a$ is a factor of $|H|$, it must be coprime to $p$. Therefore $y^a \neq e$. We further have
    \[
        (y^a)^p = (y^p)^a =e.
    \]
    We conclude that $y^a$ is of order $p$ in $G$.
\end{proof}


\begin{cor}
    Let $G$ be a finite abelian group and let $p$ be a prime that divides the order of $G$. Then Sylow $p$-subgroup of $G$ exists.
\end{cor}

\begin{proof}

\end{proof}
\begin{thm}
    Let $G$ be a finite group and let $p$ be a prime. Then Sylow $p$-subgroup of $G$ exists.
\end{thm}
\begin{proof}
    We can assume $p \vert |G|$, otherwise there is nothing to show. Let us assume $p^n \vert |G|$ but $p^{n+1} \nmid |G|$.
    We proceed by induction on $|G|$. The base case is trivial.

    If $p \mid |Z(G)|$, then we have an element $x \in Z(G)$ of order $p$. This is because $Z(G)$ is abelian, hence we can apply the previous lemma. Then if $P' \le G/\langle x \rangle$ is a Sylow $p$-subgroup of the quotient, $\pi^{-1}(P')$ will the Sylow p-subgroup of $G$.

    Assume $p \nmid |Z(G)|$. We write
    \[
        |G| = |Z(G)| + \sum_{i =1}^n |G/C_G(g_i)|.
    \]
    Here $\{g_1, \dots, g_n\}$ is a set of representatives of non-trivial conjugacy classes. Since $p \vert |G|$ and $p \nmid |Z(G)|$, we must have $p \nmid |G/C_G(g_i)|$ for some $i$.

    Let us assume $p \nmid |G/C_G(g_1)|$. Then $p^n \nmid |C_G(g_1)|$. By assumption of $g_1$ (non-trivial conjugacy class), we must have have $C_G(g_1) \neq G$, or $|C_G(g_1)| < |G|$. We apply induction hypothesis to obtain a Sylow $p$-subgroup of $|C_G(g_1)| $ of order $p^n$. This is clearly the Sylow $p$-subgroup of $G$ as well.
\end{proof}




\subsection{Sylow theorems II}

Let $S$ be the set of all Sylow $p$-subgroups. Then $|S| = n_p$ by definition. We know $S$ is not empty now. We consider the action of $G$ on $S$ by conjugation. Let $Q \in S$ and $G\cdot Q$ be the orbit of $Q$. The next few theorems explore the action of a $p$-subgroup (could be a Sylow $p$-subgroup as well) $P$ on $S$ and $G\cdot Q$.

Let us record a useful lemma here.
\begin{lem}
    Let $Q$ be a Sylow $p$-subgroup of $G$. Let $P$ be any $p$-subgroup of $G$, then we have $(N_P(Q) =) P \cap N_G(Q) = P \cap Q$.
\end{lem}

\begin{proof}
    Let $H = P \cap N_G(Q) = \{g \in P \vert gQg^{-1} = Q\}$. It is clear that $Q \cap P \subset H$. We show the reverse inclusion.

    We claim $HQ $ is a $p$-subgroup of $G$. It is straightforward to check that $HQ$ is a subgroup of $G$ and $Q$ is a normal subgroup of $H$. By the isomorphism theorem, we have
    \[
        HQ /Q \cong H/ H\cap Q.
    \]
    We conclude that $|HQ| = \frac{|H||Q|}{|H\cap Q|}$. Note that $|H|$, $|Q|$, $|H\cap Q|$ are all powers of $p$. So $HQ$ is a $p$-subgroup of $G$ containing the Sylow $p$-subgroup $Q$. We must have $HQ =Q$, that is, $H \subset Q$. Hence $H \subset P \cap Q$.
\end{proof}



\begin{prop}
    Let $G$ be a finite group and let $p$ be a prime. Then we have
    \[
        n_p \equiv 1 \mod p.
    \]
\end{prop}
\begin{proof}
    Let $P$ be a Sylow $p$-subgroup of $G$. We consider the action of $P$ on $S$ by restriction.

    We consider the partition of $S$ into $P$-orbits, say
    \[
        S = \mathcal{O}_1 \sqcup \mathcal{O}_2 \sqcup \cdots \sqcup \mathcal{O}_n.
    \]
    Of course $P \in S$ is an orbit consists of a single element. Let us just call this orbit $\mathcal{O}_1$. Then we have
    \[
        n_p=|S| = 1 + |\mathcal{O}_2| + \cdots |\mathcal{O}_n|.
    \]



    For any $ \mathcal{O}_i$ with $i \neq 1$, we have bijections
    \[
        P/ Stab_P(Q_i) \cong  \mathcal{O}_i, \quad \text{for any } P \neq Q_i \in \mathcal{O}_i
    \]
    Here $Stab_P(Q_i) = \{g \in P \vert g Q_ig^{-1} = Q_i\}$. We have $Stab_P(Q_i) = P\cap Q_i$ by the lemma. Then we see that $ p \mid |P/ Stab_P(Q_i)|$. Hence
    \[
        |S| \equiv 1 \mod p. \qedhere
    \]
\end{proof}

\begin{cor}\label{cor:SylowCor}
    We have
    \[
        | G \cdot Q | \equiv 1 \mod p.
    \]
\end{cor}

\begin{proof}
    We consider the action of $Q$ on $G\cdot Q$. Then the same argument as the preivous theorem applies.
\end{proof}


\begin{thm}
    Let $G$ be a finite group and let $p$ be a prime. Any $p$-subgroup is contained in some Sylow $p$-subgroup.
\end{thm}
\begin{proof}
    Let $P$ be a $p$-subgroup of $G$. We consider the action of $P$ on $S$. We consider the partition of $S$ by $P$-orbits, say
    \[
        S = \mathcal{O}_1 \sqcup \mathcal{O}_2 \sqcup \cdots \sqcup \mathcal{O}_n.
    \]
    We have bijections
    \[
        P/ Stab_P(Q_i) \cong \mathcal{O}_i, \quad Q_i \in \mathcal{O}_i.
    \]
    Recall $Stab_P(Q_i) = \{g \in P \vert g Q_ig^{-1} = Q_i\} = P \cap Q_i$ by the previous lemma. We see thave
    \[
        \begin{cases}
            p \mid |P/ Stab_P(Q_i)|= |\mathcal{O}_i|, & \text{if } P \not \subset Q_i; \\
            1 =|P/ Stab_P(Q_i)|= |\mathcal{O}_i|,     & \text{if } P \subset Q_i.
        \end{cases}
    \]
    Since $|S| = |\mathcal{O}_1| + \cdots + |\mathcal{O}_n|$ is $1$ mod $p$, we must have $ 1 =|P/ Stab_P(Q_i)|$ for some $i$. Hence $P$ is contained in some Sylow $p$-subgroup.
\end{proof}

\begin{thm}
    Let $G$ be a finite group and let $p$ be a prime. Any two Sylow $p$-subgroups are conjugate to each other. In other words, the action of $G$ on $S$ is transitive.
\end{thm}
\begin{proof}
    Let $P$ and $Q$ be Sylow $p$-subgroups. We consider the action of $P$ on $G\cdot Q$. By definition $G\cdot  Q = \{gQg^{-1}\vert g \in G\}$. We can then apply the same argument as the previous one thanks to Corollary~\ref{cor:SylowCor}.
\end{proof}

\begin{thm}
    Let $G$ be a finite group and let $p$ be a prime. Then we have
    \[
        n_p \mid |G|.
    \]
\end{thm}
\begin{proof}
    We know now $S$ is a single $G$-orbit. So we have a bijection
    \[
        G/Stab_G(P) \cong S, \quad\text{for any } P \in S.
    \]
    Then since $|G/Stab_G(P)|$ divides $|G|$, $n_p$ divides $|G|$.
\end{proof}


\subsection{Consequences of Sylow theorems}
We next discuss some consequences of Sylow theorems.

\begin{cor}
    Let $G$ be a finite group and let $p$ be a prime.
    \begin{enumerate}
        \item Let $P$ be a $p$-subgroup of $G$ and $Q$ be a Sylow $p$-subgroup of $G$. Then we have
              \[
                  P \subset gQg^{-1}, \text{for some } g \in G.
              \]
        \item $G$ has a unique Sylow $p$-subgroup $P$ if and only if the Sylow $p$-subgroup $P$ is normal.
    \end{enumerate}
\end{cor}

\begin{example}
    We consider the symmetric group $S_3$. There are three Sylow 2-subgroups: $\langle (12) \rangle $, $\langle (23) \rangle $, $\langle (13) \rangle $. There is only one Sylow 3-subgroup $\langle (123) \rangle \cong A_3$, which is normal.

\end{example}
\begin{example}
    Let us classify groups of order $15$ (up to isomorphism).

    Let $G$ be such a group. We know
    \[
        n_3 \equiv 1 \mod 5 \quad \text{ and } \quad n_3 \vert 15.
    \]
    We must have $n_3 =1$. So we have a unique normal Sylow $3$-subgroup $P_3$. Similarly we see that we have a unique normal Sylow $5$-subgroup $P_5$. Note that since $|P_3| =3$ and $|P_5 | = 5$ are both primes, we must have $P_3 \cong \mathbb{Z}/ 3 \mathbb{Z}$ and $P_5 \cong \mathbb{Z}/ 5 \mathbb{Z}$.

    We then make the following claims
    \begin{enumerate}
        \item $P_3 P_5$ is a subgroup of $G$
        \item We have $|P_3 P_5| = \frac{|P_3| |P_5|}{ |P_3 \cap P_5|} = 15$. (This is a special case of double cosets.)
        \item We have $P_3 P_5 = G$ for numerical reason.
        \item We have $G \cong P_3 \times P_5 \cong \mathbb{Z}/ 15\mathbb{Z}$.
    \end{enumerate}

    So we have only one group of order $15$.
\end{example}




\subsection{Semi-direct products}

\begin{defi}
    Let $H$ and $K$ be two groups. Let $\phi : K \rightarrow Aut(H)$ be a group homomorphism. We define a binary operation on $H \times K$ by
    \[
        (h_1, k_1) \cdot (h_2, k_2) = (h_1 \phi(k_1)(h_2), k_1k_2).
    \]
\end{defi}

\begin{thm}
    The binary operation defines a group structure on the set $H \times G$. We denote this group by $H \rtimes_{\phi} K$ (or simply $H \rtimes K$). This is called the semi-direct product of $H$ and $K$ with respect to $\phi$.
\end{thm}
\begin{proof}
    Intuitively, we want to think of $\phi(k_1)(h_2)$ as $k_1 h_2 k_1^{-1}$.
\end{proof}

\begin{remark}
    We could have $H \rtimes_{\phi} K \cong H \rtimes_{\psi} K$ for two different group homomorphisms $K \rightarrow Aut(H)$. This will be precise in the next Proposition.
\end{remark}

\begin{prop}Let $H \rtimes_{\phi} K$ be the semi-direct product of $H$ and $K$ with respect to $\phi$.
    \begin{enumerate}
        \item $|H \rtimes_{\phi} K| = |H| |K|$.
        \item $\{(h,e) \vert h \in H\}$ is  a normal subgroup of $H \rtimes_{\phi} K$ isomorphic to $H$. We often just identify this subgroup with $H$.
        \item $\{(e,k) \vert k \in K\}$ is  a subgroup of $H \rtimes_{\phi} K$ isomorphic to $K$. We often just identify this subgroup with $K$.
        \item $H \cap K = \{e\}$.
        \item For any $k \in K$ and $h \in H$, we have $k h k^{-1} = \phi(k)(h)$.
    \end{enumerate}
\end{prop}

\begin{proof}

\end{proof}
\begin{example}
    \begin{enumerate}
        \item   Let $\phi: K \rightarrow Aut(H)$ be the trivial group homomorphism. Then $H \rtimes_\phi K \cong H \times K$.
        \item Let $G$ be a group. We consider the permutation map $\phi: S_n \rightarrow Aut(G^n)$. Then the semi-direct product $(G^n) \rtimes_\phi S_n$ is called the wreath product of $G$ by $S_n$, and often denoted by $G \wr S_n$.

              The multiplication behaves as follows
              \[
                  ((g_i), \sigma) \cdot ((h_i), \tau) = ((g_i h_{\sigma(i)}), \sigma\tau).
              \]
    \end{enumerate}
\end{example}
\begin{prop}
    Let $G$ be a group with two subgroups $H$ and $K$. Assume
    \begin{enumerate}
        \item $H$ is normal in $G$,
        \item $H \cap K = \{e\}$,
        \item $HK = G$.
    \end{enumerate}
    Then we have $G \cong H \rtimes_\phi K$. Here $\phi: K \rightarrow Aut(H)$ is given by $k \mapsto (h \mapsto k h k^{-1})$.
\end{prop}
\begin{proof}

\end{proof}


\begin{prop}
    There are exactly two groups (up to isomorphism) of order $6$.
\end{prop}
\begin{proof}
    We know we have two such groups $\mathbb{Z}/6\mathbb{Z}$ and $S_3$.

    Let $G$ be such a group. We know
    \[
        n_3 \equiv 1 \mod 3 \quad \text{ and } \quad n_3 \vert 6.
    \]
    We must have $n_3 =1$. So we have a unique normal Sylow $3$-subgroup $P_3 \cong \mathbb{Z}/3\mathbb{Z}$.

    We similarly know
    \[
        n_2 \equiv 1 \mod 2 \quad \text{ and } \quad n_2 \vert 6.
    \]
    We have two cases for $n_2$. We have either $n_2 =1$ or $n_2 = 3$.

    If $n_2 =1$, then very much similar to the previous example, we have $G \cong \mathbb{Z}/2\mathbb{Z} \times  \mathbb{Z}/3\mathbb{Z} \cong \mathbb{Z}/6\mathbb{Z}$.

    Assume $n_2 =3$ now. Let $P_2$ be any one of the three Sylow $2$-subgroups. We still have following claims
    \begin{enumerate}
        \item $P_2P_3$ is a groups.
        \item We have $|P_2 P_3| = \frac{|P_2| |P_3|}{ |P_2 \cap P_3|} = 6$. (This is a special case of double cosets.)
        \item We have $P_2 P_3 = G$ for numerical reason.
    \end{enumerate}
    However, $G \not \cong P_2 \times P_3$. Let us consider the conjugation action of $P_2$ on $P_3$. This is well-defined, since $P_3$ is normal. (Recall from Midterm, if $P_2$ is also normal, then this action is trivial.) So now we have a group homomorphism $\phi: P_2 \rightarrow Aut (P_3) = Aut (\mathbb{Z}/3\mathbb{Z}) \cong \mathbb{Z}/2\mathbb{Z}$.

    We further divide into two cases.
    \begin{enumerate}
        \item If $\phi$ is the trivial group homomorphis, then we have $G \cong P_2 \times P_3$. We actually have a contradiction here, or we can relax the assumption. In any case, we have considered this case before.
        \item If $\phi$ is the non-trivial group homomorphism, this means $x y x^{-1}  =y ^2$, there $x$ is the generator $P_2$ and $y$ is the generator of $P_3$.

              In this case, we have a group homomorphism $P_3 \rtimes_\phi P_2 \to G$. We see that this is an isomorphism by cardinality reason.
    \end{enumerate}
\end{proof}


\subsection{Applications of Sylow's theorems}

\begin{defi}
    A simple group is a nontrivial group whose only normal subgroups are the trivial group and the group itself.
\end{defi}

\begin{example}
    Let $G$ be a group of order $132$. Then $G$ can not be simple.

    We have $132 = 11 \times 2^2 \times 3$. We have
    \[
        n_3 \equiv 1 \mod 3 \quad \text{and} \quad  n_3 \mid 132
    \]
    \[
        n_2 \equiv 1 \mod 2 \quad \text{and} \quad  n_2 \mid 132
    \]
    \[
        n_{11} \equiv 1 \mod 11 \quad \text{and} \quad  n_{11} \mid 132
    \]

    Assume the contrary that $G$ is simple. Then we must have $n_{11} = 12$. Note that two Sylow $11$-subgroups intersection trivially.  So there are $12 \times 10 = 120$ elements of order $11$.

    Now we look at $n_3 \in \{1, 4, 22\}$. We know $n_3 \neq 1$ by assumption. If $n_3 = 4$, then we have $ 4 \times 2 =8$ elements of order $3$. So the Sylow $2$-subgroup must be normal. If $n_3 = 22$, then we clearly have too many elements beyond the cardinality of $G$.
\end{example}

\begin{example}
    Let $G$ be a group of order $12 = 2^2 \times 3$. Then we claim either $G$ has a normal Sylow $3$-subgroup or $G$ has a normal Sylow $2$-subgroup. So this group can not be simple.

    Assume $n_3 \neq 1$. Then we know
    \[
        n_3 \equiv 1 \mod 3 \quad n_3 \mid 12.
    \]
    We can only have $n_3 =4$. Note that different Sylow $3$-subgroups have trivial intersections. So the union of the four Sylow $3$-subgroups contains $9$ elements. We have only $3 + 1$ elements left for the Sylow $2$-subgroups. It has to be normal.




    Let us determine the group in this case. Let $S = \{Q_1, Q_2, Q_3, Q_4\}$ be the set of Sylow $3$-subgroups. We have the conjugation action of $G$ on $S$, hence a group homomorphism
    \[
        G \rightarrow S_4.
    \]
    Recall $Stab_{Q_i} (Q_j) = Q_i \cap Q_j$. Therefore the image of $Q_i$ consisting of 3-cycles fixing $Q_i$ while permuting the other three subgroups.

    There are exactly four Sylow $3$-subgroups contained in $S_4$, and they generate $A_4 = H$. Then since $|A_4| = 12$, we have $G \cong A_4$ for cardinality reason.
\end{example}

Let us try to classify groups of order $12$.

\begin{lem}
    Let $p$ be a prime. Then any group $G$ of order $p^2$ must be abelian.

    Moreover, we have either $G \cong \mathbb{Z}/p^2 \mathbb{Z}$ or $G \cong \mathbb{Z}/p\mathbb{Z} \times \mathbb{Z}/p\mathbb{Z}$.
\end{lem}

\begin{proof}
    We know $G$ must have non-trivial center. Let $x \in Z(G)$ be a non-trivial elements, and let $H = \langle x \rangle$. If $H = G$, we are done. In this case we have $G \cong  \mathbb{Z}/p^2 \mathbb{Z}$.

    Otherwise we have $|H| = p$ and $H \cong \mathbb{Z}/p\mathbb{Z}$. Let $y \in G$ and $ y\notin H$. And consider $K = \langle y \rangle$. If $K  =G$, we done again. In this case we have again $G \cong  \mathbb{Z}/p^2 \mathbb{Z}$.

    Now we are left with the case $H \cong \mathbb{Z}/p\mathbb{Z}$ and $K \cong \mathbb{Z}/p\mathbb{Z}$. Note that $H \cap K$ is trivial. We also have $xy = yx$ for any $x \in H$ and $y \in K$, since any group homomorphsim $K \rightarrow Aut(H)$ is trivial ($|Aut(H)|= p-1$). Therefore the map $H \times K \rightarrow G$, $(x, y) \mapsto xy$ is a group isomorphism.
\end{proof}

\begin{lem}
    Let $G$ be a group (potentially infinite) such that $G/Z(G)$ is cyclic (including the trivial case). Then $G$ is abelian.

    In other words, $G/Z(G)$ can be not a non-trivial cyclic group.
\end{lem}
\begin{proof}
    HW 5.
\end{proof}

\begin{example}
    Let us now classify groups of order $12$. Let $G$ be such a group. We already know that if $n_3 \neq 1$ then we have $G \cong A_4$. We assume $n_3 = 1$ now, and let $P_3$ be the Sylow $3$-subgroup.

    Assume $n_2 =1$. Let $P_4$ be the unique normal Sylow $2$-subgroup. We know $P_4 \cong \mathbb{Z}/2^2 \mathbb{Z}$ or $P_4 \cong \mathbb{Z}/2\mathbb{Z} \times \mathbb{Z}/2\mathbb{Z}$.  Then we have $G \cong P_4 \times P_3$, since $P_4 \cap P_3 = \{e\}$ and both of them are normal.



    Assume $n_2 \neq 1$. Let $P_4$ be a Sylow $2$-subgroup. We know $P_4P_3 = G$. So we need to determine the multiplication of $G$, which is essentially the group homomorphism $P_4 \rightarrow Aut(P_3) \cong \mathbb{Z}/2\mathbb{Z}$.
    \begin{enumerate}
        \item Assume $P_4 \cong \mathbb{Z}/2^2 \mathbb{Z}$. We consider group homomorphisms $\phi: P_4 \rightarrow Aut(P_3) \cong \mathbb{Z}/2\mathbb{Z}$. There are only two of them, denoted by $\phi_1$ and $\phi_2$, where $\phi_1$ is the trivial one. We have $P_3 \rtimes_{\phi_1} P_4 \cong \mathbb{Z}/3 \mathbb{Z} \times \mathbb{Z}/2^2 \mathbb{Z}$ and $\mathbb{Z}/3 \mathbb{Z} \rtimes_{\phi_2}\mathbb{Z}/2^2 \mathbb{Z}$.
        \item Assume $P_4 \cong \mathbb{Z}/2 \mathbb{Z} \times   \mathbb{Z}/2  \mathbb{Z}$. We consider group homomorphisms $\phi: P_4 \rightarrow Aut(P_3) \cong \mathbb{Z}/2\mathbb{Z}$. There are four of them $\phi_1$, $\phi_2$, $\phi_3$, $\phi_4$. Here $\phi_1$ is the trivial one. We assume $\phi_2$ maps $(a,b)$ to $a$, and $\phi_2$ maps $(a,b)$ to $b$, and $\phi_4$ maps $(a,b)$ to $a +b$.

              Then we have $P_3 \rtimes_{\phi_1} P_4 \cong \mathbb{Z}/3 \mathbb{Z} \times \mathbb{Z}/2  \mathbb{Z} \times \mathbb{Z}/2  \mathbb{Z}$. Then we can check by direct computation that   $P_3 \rtimes_{\phi_2} P_4 \cong  P_3 \rtimes_{\phi_3} P_4 \cong  P_3 \rtimes_{\phi_4} P_4$. We see that $P_3 \rtimes_{\phi_2} P_4 \cong G' \times \mathbb{Z}/2\mathbb{Z}$ for a non-abelian group $G'$ of order $6$. We see that $G' \cong S_3$ by our earlier result. We conclude that $P_3 \rtimes_{\phi_2} P_4 \cong S_3 \times \mathbb{Z}/2\mathbb{Z}$.
    \end{enumerate}

    Now we can conclude that there are $5$ groups of order $12$, up to isomorphism.
\end{example}



\subsection{Solvable groups}

\begin{definition}
    Let $G$ be a group.
    \begin{enumerate}
        \item A {\em (normal) tower/series} of $G$ is a sequence of subgroups
              \[
                  G = G_0 \supset G_1 \supset G_2 \supset \cdots \supset G_m  (= \{e\})
              \]
              such that $G_{i+1}$ is a {\em (normal)} subgroup of $G_i$ (not necessarily of $G$). We have the {\em subquotient/factor} groups $G_i/ G_{i+1}$. The normal tower is called abelian (resp. cyclic), if each factor group $G_i/ G_{i+1}$ is abelian (resp. cyclic).

        \item A {\em refinement} of a given tower is a tower obtained by inserting a finite number of subgroups in the given tower.

        \item Let
              \[
                  \begin{split}
                      G = H_0 \supset H_1 \supset H_2 \supset \cdots \supset H_n = \{e\},\\
                      G = G_0 \supset G_1 \supset G_2 \supset \cdots \supset G_m = \{e\}
                  \end{split}
              \]
              be normal towers. Two normal towers are called {\em equivalent} if $m = n$ and up to permutation of indices $i \mapsto i'$, we have
              \[
                  G_i/ G_{i+1} \cong H_{i'}/ H_{i'+1}, \qquad \text{ for all } i.
              \]
    \end{enumerate}
\end{definition}

\begin{lem}
    Let $G$ be a finite group. An abelian tower of $G$ admits a cyclic refinement.
\end{lem}
\begin{proof}

\end{proof}

\begin{definition}
    A normal tower
    \[
        G = G_0 \supset G_1 \supset G_2 \supset \cdots \supset G_m = \{e\}
    \]
    is called a {\em composition series} of $G$ is each factor group $G_i/G_{i+1}$ is simple. The factor groups are called {\em composition factors} of $G$. Note that this is NOT well-defined yet.

    (Recall a group $H$ is called simple if $H\neq \{e\}$ and it does not contain any other normal subgroups besides $\{e\}$ and $H$.)
\end{definition}

\begin{remark}
    The composition series always exist for a finite group $G$. The group $\mathbb{Z}$ has no composition series.
\end{remark}
\begin{example}
    \begin{enumerate}
        \item Let $G= \BZ/6\BZ$. We have two equivalent normal towers
              \[
                  G \supset \BZ/3\BZ \supset \{e\}, \quad G \supset \BZ/2\BZ \supset \{e\}.
              \]
        \item The two groups $\Z/4\Z$ and $\Z/2\Z \times \Z/2\Z$ have the same composition factors, while non-isomorphic.
        \item
              A group $G$ is called {\em solvable} if it admits a normal tower
              \[
                  G = G_0 \supset G_1 \supset \cdots \supset G_m = \{e\}
              \]
              such that $G_{i}/G_{i+1}$ is abelian.

              We claim $S_3$ is solvable. We actually have the normal tower
              \[
                  S_3 \supset A_3 \supset \{e\}.
              \]
        \item $S_5$ is not solvable (Google this). This plays a VERY important role in Galois theory.
    \end{enumerate}
\end{example}



\begin{lem}
    Let $G$ be a group. The commutator subgroup $G^{(1)} = [G, G]$ of $G$ is defined to be the subgroup generated by $[a,b]= aba^{-1}b^{-1}$ for all $a, b \in G$. Then $G^{(1)}$ is normal in $G$. In particular, any group homomorphism from $G$ to an abelian group factors through $G/ [G, G]$.


    We similarly define $G^{(i+1)} = [G^{(i)}, G^{(i)}]$.
\end{lem}

\begin{proof}
    Let $u \in [G, G]$. Then
    \[
        gug^{-1} = u \cdot u^{-1} gug^{-1}  = u \cdot [u^{-1}, g] \in[G, G].
    \]
    This shows the normality.  The rest follows from the universal property of the quotient.
\end{proof}
We often write $G^{(0)} = G$.

\begin{prop}
    A group $G$ is solvable if and only if $G^{(n)} = \{e\}$ for some $n$.
\end{prop}

\begin{proof}
    We assume $G^{(n)} = \{e\}$ for some $n$. Since $G/ [G,G]$ is always abelian, the claim follows.

    Now we prove the other direction. Let
    \[
        G = H_0 \supset H_1 \supset \cdots \supset H_m = \{e\}
    \]
    be a normal tower with abelian factor groups. Since $H_i/ H_{i+1}$ is abelian, we must have
    \[
        H_{i+1} \supset [H_i, H_i].
    \]
    We then claim $G^{(i)} = [G^{(i-1)}, G^{(i-1)}] \subset [H_{(i-1)}, H_{(i-1)}] \subset H_{i}$ by induction. This is immediate, since $G^{(0)} = H_0$.
\end{proof}




\section{Nilpotent groups}

\begin{defi}
    \begin{enumerate}
        \item For any (finite or infinite) group $G$ we define the following subgroups inductively:
              \[
                  Z_0 (G) =1, Z_1 (G) = Z(G)
              \]
              and $Z_{i+1}(G)$ is the subgroup $\pi^{-1} (Z (G/ Z_i(G)))$ for the canonical quotient $\pi: G \rightarrow G/Z_i(G)$.

              The chain of (normal) subgroups
              \[
                  Z_0 \le Z_1 \le Z_2 \le \cdots
              \]
              is called the upper central series of $G$.
        \item A group $G$ is called nilpotent if $Z_n(G) =G$ for some $n$. The smallest such $n$ is called the nilpotence class of $G$.
    \end{enumerate}
\end{defi}

\begin{cor}If $G$ is nilpotent, then $G$ is solvable.
\end{cor}

\begin{example}
    \begin{enumerate}
        \item If $G$ is abelian, then $G$ is nilpotent.
        \item We have $Z_n (S_3) = \{e\}$ for any $n$. So $S_3$ is not nilpotent. So $S_3$ is solvable, but not nilpotent.
    \end{enumerate}
\end{example}


\begin{remark}
    There are various equivalent characterizations of nilpotent groups.
\end{remark}
\begin{lem}
    Let $G$ be a finite $p$-group for some prime $p$. Then $G$ is nilpotent.
\end{lem}
\begin{proof}

\end{proof}



% \begin{prop}
% Let $G$ be a nilpotent group. Then any subgroup or quotient group of $G$ is also nilpotent. 
% \end{prop}

% \begin{proof}
% Let $H \le G$. We first have 
% \[
% Z_0(G) \le Z_1(G) \le \cdots \le Z_n(G) = G.
% \]
% \end{proof}

\begin{thm}
    Let $G$ be a finite group of order $p_1^{n_1} \cdots p_k^{n_k}$ with primes $p_i$ and $n_i >0$. Let $P_i$ be a Sylow $p_i$-subgroup of $G$. Then the following are equivalent:
    \begin{enumerate}
        \item $G$ is nilpotent;
        \item if $H$ is a proper subgroup of $G$, then $H$ is a proper subgroup of $N_G(H)$;
        \item every Sylow $p_i$-subgroup is normal;
        \item $G \cong P_1 \times P_2 \times \cdots \times P_k$.
    \end{enumerate}
\end{thm}

\begin{proof}
    We show $(1) \implies (2)$. We proceed on induction of $|G|$. The base case is vacuous.

    We know $Z(G) \neq \{e \}$. We clearly have $H Z(G) \subset N_G(H)$. We can assume $Z(G) \subset H$, otherwise, we are done. We consider the quotients $H / Z(G) \rightarrow G /Z(G)$. Then $H/ Z(G)$ is a proper subgroup of $G /Z(G)$. Let $K / Z(G)$ be the normalizer of $H/ Z(G)$ in $G /Z(G)$. We know $H/ Z(G)$ is a proper subgroup of $K / Z(G)$ by induction hypothesis. Hence $H$ is a proper subgroup of $K$. We claim $K \subset  N_G(H)$. For any $ h \in H$ and $k \in K$, we have $k h k^{-1} Z(G) \subset HZ(G) = H$. The claim follows.

    We show $(2) \implies (3)$. Let $N = N_G(P_i)$. We know $P_i$ is a normal subgroup of $N$, and the unique Sylow $p_i$-subgroup of $N$. Let $H = N_G(N)$. Then we claim $H = N$. We clearly have $N \subset H$. On the other hand, for any $h \in H$, we have $hNh^{-1} = N$ by definition. This means $hP_ih^{-1} \subset N$ as well. But since $ hP_ih^{-1}$ is a Sylow $p_i$-subgroup of $N$, we must have $ hP_ih^{-1} = P_i$. Therefore $h \in N = N_G(P_i)$.  This proves the claim. Then by $(1)$, we see that $N = N_G(N) = G$.

    We show $(3) \implies (4)$. We have shown before that $P_1P_2 \cong P_1 \times P_2$. Now $P_1P_2$ and $P_3$ are  normal subgroups of $G$ such that $P_1P_2 \cap P_3 = \{e\}$. Then we have $P_1P_2P_3 \cong P_1 \times P_2 \times P_3$. We then proceed by induction.

    Finally, we show $(4) \implies (1)$. We know $P_i$ has nontrivial center. Therefore $P_1 \times P_2 \times \cdots \times P_k$ has non-trivial centers. We can repeat this argument for the quotient $G/Z(G)$ to show $Z_1 (G) \neq Z(G)$. Since $G$ is finite, we eventualy must have $Z_n(G) = G$ for some $n$.
\end{proof}


\begin{prop}
    Let $G$ be a finite group. Let $H$ be a normal subgroup of $G$ and $P$ be a Sylow $p$-subgroup of $H$. Then $G = H N_G(P)$.
\end{prop}

\begin{proof}
    For any $g \in G$, since $H$ is normal, we have $g P g^{-1} \subset H$. Then we apply the Sylow theorem to the group $H$, we see that $g P g^{-1} = h P h^{-1}$ for some $h \in H$. In other words, we have $h^{-1}g \in N_G(P)$. Hence $g \in HN_G(P)$. Therefore $G=HN_G(P)$.
\end{proof}

\begin{defi}
    Let $G$ be a group. A proper subgroup $M$ of $G$ is called maximal if whenever $H \le H \le G$, then either $H= M$ or $M= G$.
\end{defi}
\begin{prop}
    Let $G$ be a finite group. Then $G$ is nilpotent if and only if all maximal subgroups of $G$ is normal.
\end{prop}

\begin{proof}
    Let $M$ be a maximal subgroup of $G$. Then the Theorem, we know $M$ is a proper subgroup of $N_G(M)$. We must have $N_G(M) = G$. Hence $M$ is normal in $G$.

    For reverse implication, we show every Sylow $p$-subgroup is normal (for any prime $p$). Let $P$ be a Sylow $p$-subgroup. Assume the contrary that $P$ is not normal in $G$. Then we can find a maximal subgroup $M$ containing $N_G(P)$ (since $G$ is finite). Then we see that $M$ is normal in $G$ by assumption. Then $MN_G(P) = G$ by the lemma. But $N_G(P) \subset M$, hence $M =  MN_G(P) = G$. We have a contradiction.
\end{proof}

\subsection{Inverse limits}
We consider a sequence of groups $\{G_n\}_{n=1}^{\infty}$ together with group homomorphisms $f_n: G_n \rightarrow G_{n-1}$. We define the inverse limit $\varprojlim G_i$ of of the sequence of groups as follows. As a set, we have
\[
    \varprojlim G_i = \{(x_i) \vert x_i \in G_i, f_i(x_i)= x_{i-1}\}.
\]
We then define the multiplication on $\varprojlim G_i$ by
\[
    (x_i) \cdot (y_i) = (x_iy_i).
\]

\begin{prop}
    $\varprojlim G_i$ is a group with the multiplication defined above.
\end{prop}
\begin{proof}

\end{proof}
\begin{example}
    Let $G_n = \mathbb{Z}/p^n \mathbb{Z}$ for $n \ge 1$ and let $f_n : G_n \rightarrow G_{n-1}$ be the canonical quotient. Then the group $\varprojlim G_n$ is called $p$-adic integers, denoted by $\mathbb{Z}_p$. We often only consider the case when $p$ is a prime. We will see this is actually a ring in the future semester.

    An element in $\mathbb{Z}_p$ is a sequence $(x_n)$. For example in $\mathbb{Z}_3$, we have
    \[
        (0, 2 \times 3 + 0, 3^2 + 2 \times 3 + 0, \dots, )
    \]
    Equivalently, we can write $(x_n) \in \mathbb{Z}_p$ as $ \sum_{i = 0}^\infty a_i p^i$ where $ 0\le a_i < p$. Then we have  $x_n = \sum_{i = 0}^{n-1} a_n p^n$.
\end{example}